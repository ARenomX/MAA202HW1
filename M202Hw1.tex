\documentclass{article}

\usepackage{amsmath}
\usepackage{amssymb}
\usepackage{textcmds}
\usepackage{graphicx}
\usepackage{subcaption}
\usepackage{float}
\usepackage{hhline}

\title{MAA202: Analysis\\Homework I}

\date{4$^{th}$ November 2019}
\author{Ma\"elys Solal\\Alexandre Hirsch\\Andr\'e Renom}

\begin{document}


	\pagenumbering{gobble}
	\maketitle

	%\tableofcontents
	%\listoffigures
	\newpage
	\pagenumbering{arabic}
	%\renewcommand{\thesection}{\Roman{section}} %To set Section numbering to Roman

\section{Exercise 1}
\subsection{}
In order to show that

\subsection{}
We want to show that for a cool set $X$, and $\alpha,\beta \geq 0, \, \alpha X + \beta X = (\alpha + \beta )X$\\
We will procede by double inculsion, let us therefore first how that $\alpha X + \beta X \subset (\alpha + \beta )X$. We will take $a,b \in X$. Then $a\alpha + b\beta \in  \alpha X + \beta X$.
\begin{align*}
	a\alpha + b\beta = (\alpha + \beta )\left( \frac{\alpha}{\alpha + \beta}a +  \frac{\beta}{\alpha + \beta}b \right)
\end{align}
We have that 

 
%%% Table Format %%%

%\begin{figure}[H]
%\begin{tabular}{ |p{2cm}||p{1.7cm}||p{3.8cm}||p{3.1cm}|}
%\hline
%\multicolumn{4}{|c|}{Table of Results with Uncertainties} \\
%\hline
%Distance (m) & Time (ns) & Uncertainty (Distance) & Uncertainty (time) \\
%\hline
%\hline
%\end{tabular}
%\caption{Experimental Results Table With Uncertainties}
%\end{figure}


%%% Image Format %%%

%\begin{figure}[H]
%	\centering
%	\includegraphics[width=0.5\linewidth]{circuit.png}
%	\caption{Experiment 1: Circuit Set-Up}
%\end{figure}

%%% Equation Format %%%

%\begin{align*}
%	a &= 10
%	bcde &= 1
%\end{align*}


\end{document}