\documentclass{article}

\usepackage{amsmath}
\usepackage{amssymb}
\usepackage{textcmds}
\usepackage{graphicx}
\usepackage{subcaption}
\usepackage{float}
\usepackage{hhline}

\title{MAA202: Analysis\\Homework I}

\date{4$^{th}$ November 2019}
\author{Ma\"elys Solal\\Alexandre Hirsch\\Andr\'e Renom}

\begin{document}


	\pagenumbering{gobble}
	\maketitle

	%\tableofcontents
	%\listoffigures
	\newpage
	\pagenumbering{arabic}
	%\renewcommand{\thesection}{\Roman{section}} %To set Section numbering to Roman

\section{Exercise 1}
\subsection{}

Let $B = \overline{B}(0,1) = \{x \in E \ | \ ||x|| \leq 1\}$ be the closed unit ball for the norm $||\cdot||$. Let us show that $B$ is cool. \\

Let us prove that $B$ is convex. \\
Take $x, y \in B$ and $t \in [0, 1]$. We want to prove that $tx + (1-t)y \in B$. To do so, we prove that $||tx + (1-t)y|| \leq 1$. 
\begin{align*}
 	||tx + (1-t)y|| &\leq ||tx|| + ||(1-t)y|| \quad \text{by the triangle inequality} \\
	&= |t|\cdot ||x|| + |1-t|\cdot ||y|| \quad \text{by homogeneity} \\
	&\leq t + (1-t) \quad \text{as} \  ||x|| \leq 1 \  \text{and} \  ||y|| \leq 1 \\
	&= 1 
\end{align*}
Which finally proves that B is convex. \\

We now prove that $B$ is bounded. \\
Take $x \in B$ then $||x|| \leq 1$ by definition of $B$. This proves that $B$ is bounded by $1$. \\

We now prove that $B$ is symmetric with respect to $0$. \\
Take $x \in B$ then $||x|| \leq 1$ by definition of $B$ and thus $|-1|\cdot||x|| \leq 1$ and by homogeneity $||-x|| \leq 1$ which proves $-x \in B$. \\
We have therefore proved that $B$ is symmetric with respect to $0$. \\

We finally prove that $0 \in \mathring{B}$ \\
Let $\widetilde{B}=B(0,1) = \{x \ | \  ||x|| < 1 \}$ be the open unit ball. We know $\widetilde{B} \subset B$ and $\widetilde{B}$ is open. As $\mathring{B}$ is the union of all open sets of $B$, we get $\widetilde{B} \subset \mathring{B}$. \\
By definition, $0 \in \widetilde{B}$ and hence $0 \in \mathring{B}$. 

\noindent Finally, we have proved that $B$ is cool.


\subsection{}
We want to show that for a cool set $X$, and $\alpha,\beta \geq 0, \, \alpha X + \beta X = (\alpha + \beta )X$. 
We will procede by double inclusion. \\

Let us first show that $\alpha X + \beta X \subseteq (\alpha + \beta )X$. That is, given $a,b \in X$, then $a\alpha + b\beta \in  \alpha X + \beta X$.
\begin{align*}
	a\alpha + b\beta = (\alpha + \beta )\left( \frac{\alpha}{\alpha + \beta}a +  \frac{\beta}{\alpha + \beta}b \right)
\end{align*}
We have that $t := \frac{\alpha}{\alpha + \beta} \in [0,1]$, and that $\frac{\alpha}{\alpha + \beta} = t-1$. We therefore have:
\begin{align*}
	a\alpha + b\beta = (\alpha + \beta )\big(at + b(t-1)\big)
\end{align*}
By the definition of a cool set, we have that $X$ is convex, and hence for $a,b \in X$, we have $c:=ta + (t-1)b \in X$. Therefore:
\begin{align*}
	a\alpha + b\beta = (\alpha + \beta )c \in (\alpha + \beta )X
\end{align*}
We therefore have that $\alpha X + \beta X \subset (\alpha + \beta )X$.\\We now want to show that $\alpha X + \beta X \supset (\alpha + \beta )X$.\\ We take $x \in (\alpha + \beta )X$. Then there exists $y \in X$ such that:
\begin{align*}
	x &= (\alpha + \beta )y\\
	&= \alpha y + \beta y \in \alpha X + \beta X
\end{align*}
We therefore have that $\alpha X + \beta X \supset (\alpha + \beta )X$.\\ This concludes the proof that $\alpha X + \beta X = (\alpha + \beta )X$.

 
%%% Table Format %%%

%\begin{figure}[H]
%\begin{tabular}{ |p{2cm}||p{1.7cm}||p{3.8cm}||p{3.1cm}|}
%\hline
%\multicolumn{4}{|c|}{Table of Results with Uncertainties} \\
%\hline
%Distance (m) & Time (ns) & Uncertainty (Distance) & Uncertainty (time) \\
%\hline
%\hline
%\end{tabular}
%\caption{Experimental Results Table With Uncertainties}
%\end{figure}


%%% Image Format %%%

%\begin{figure}[H]
%	\centering
%	\includegraphics[width=0.5\linewidth]{circuit.png}
%	\caption{Experiment 1: Circuit Set-Up}
%\end{figure}

%%% Equation Format %%%

%\begin{align*}
%	a &= 10
%	bcde &= 1
%\end{align*}


\end{document}