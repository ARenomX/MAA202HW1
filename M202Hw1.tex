\documentclass{article}

\usepackage{amsmath}
\usepackage{amssymb}
\usepackage{textcmds}
\usepackage{graphicx}
\usepackage{subcaption}
\usepackage{float}
\usepackage{hhline}

\usepackage{geometry}
\geometry{margin=3cm}

\title{MAA202: Analysis\\Homework I}

\date{4$^{th}$ November 2019}
\author{Alexandre Hirsch, Andr\'e Renom, Ma\"elys Solal}

\begin{document}


	\pagenumbering{gobble}
	\maketitle

	\tableofcontents
	%\listoffigures
	\newpage
	\pagenumbering{arabic}
	\renewcommand{\thesubsubsection}{\thesubsection.\alph{subsubsection}}
	%\renewcommand{\thesection}{\Roman{section}} %To set Section numbering to Roman

\section{Exercise 1}
\subsection{} %Exercise 1.1

Let $B = \bar{B}(0,1) = \{x \in E \ | \ ||x|| \leq 1\}$ be the closed unit ball for the norm $||\cdot||$. Let us show that $B$ is cool. \\

\noindent Let us prove that $B$ is convex. \\
Take $x, y \in B$ and $t \in [0, 1]$. We want to prove that $tx + (1-t)y \in B$. To do so, we prove that $||tx + (1-t)y|| \leq 1$. 
\begin{align*}
 	||tx + (1-t)y|| &\leq ||tx|| + ||(1-t)y|| \quad \text{by the triangle inequality} \\
	&= |t|\cdot ||x|| + |1-t|\cdot ||y|| \quad \text{by homogeneity} \\
	&\leq t + (1-t) \quad \text{as} \  ||x|| \leq 1 \  \text{and} \  ||y|| \leq 1 \\
	&= 1 
\end{align*}
Which finally proves that B is convex. \\

\noindent We now prove that $B$ is bounded. \\
Take $x \in B$ then $||x|| \leq 1$ by definition of $B$. This proves that $B$ is bounded by $1$. \\

\noindent We now prove that $B$ is symmetric with respect to $0$. \\
Take $x \in B$ then $||x|| \leq 1$ by definition of $B$ and thus $|-1|\cdot||x|| \leq 1$ and by homogeneity $||-x|| \leq 1$ which proves $-x \in B$. \\
We have therefore proved that $B$ is symmetric with respect to $0$. \\

\noindent We finally prove that $0 \in \mathring{B}$ \\
Let $\widetilde{B}=B(0,1) = \{x \ | \  ||x|| < 1 \}$ be the open unit ball. We know $\widetilde{B} \subset B$ and $\widetilde{B}$ is open. As $\mathring{B}$ is the union of all open sets contained in $B$, we get $\widetilde{B} \subset \mathring{B}$. \\
By definition, $0 \in \widetilde{B}$ and hence $0 \in \mathring{B}$. \\

\noindent Finally, we have proved that $B$ is cool. \\



\subsection{}%Exercise 1.2
We want to show that for a cool set $X$, and $\alpha,\beta \geq 0, \, \alpha X + \beta X = (\alpha + \beta )X$. We will proceed by double inclusion.\\

\noindent Let us first show that $\alpha X + \beta X \subseteq (\alpha + \beta )X$. \\
Take $a,b \in X$. Then $a\alpha + b\beta \in  \alpha X + \beta X$. We then write:
\begin{align*}
	a\alpha + b\beta = (\alpha + \beta )\left( \frac{\alpha}{\alpha + \beta}a +  \frac{\beta}{\alpha + \beta}b \right)
\end{align*}
We define $t := \frac{\alpha}{\alpha + \beta} \in [0,1]$, and thus $\frac{\beta}{\alpha + \beta} = 1-t$. We therefore have:
\begin{align*}
	a\alpha + b\beta = (\alpha + \beta )\big(at + b(1-t)\big)
\end{align*}
As $X$ is a cool set, it is convex. Hence for $a,b \in X$, we have $c:=ta + (1-t)b \in X$. Therefore:
\begin{align*}
	a\alpha + b\beta = (\alpha + \beta )c \in (\alpha + \beta )X
\end{align*}
We have therefore proved that $\alpha X + \beta X \subseteq (\alpha + \beta )X$.\\

\noindent We now want to show that $\alpha X + \beta X \supseteq (\alpha + \beta )X$.\\ 
Take $x \in (\alpha + \beta )X$. Then there exists $y \in X$ such that:
\begin{align*}
	x &= (\alpha + \beta )y\\
	&= \alpha y + \beta y \in \alpha X + \beta X
\end{align*}
We therefore have proved that $\alpha X + \beta X \supseteq (\alpha + \beta )X$.\\ 

\noindent Finally, by double inclusion, we have proved that $\alpha X + \beta X = (\alpha + \beta )X$. \\



\subsection{}%Exercise 1.3
We now define a function on E by setting for every $x \in E$, $$N_X(x) = \inf\{|\alpha| \ | \ x \in \alpha X \}$$
We now show that $N_X$ is well-defined and that it defines a norm on E. 

\subsubsection{}%Exercise 1.3.a

We want to show that for each $x \in E$, the set $N := \{\alpha \,|\, x \in \alpha X\}$ is not empty. \\

\noindent We start from the fact that $0 \in \mathring{X}$. \\
Then, there exists an $r > 0$ such that $B(0,r) \subset X$. \\
We define $\alpha := \frac{2||x||}{r}$. We know that $B(0,\alpha r) \subset \alpha X$. Since $B(0, \alpha r) = B(0, 2||x||)$, we have that $x \in \alpha X$, and hence $\alpha \in N$. \\

\noindent We have hence proved that $N$ isn't empty.

\subsubsection{}%Exercise 1.3.b

We want to show $N$ is homogeneous, that is $N_X(\lambda x) = |\lambda|N_X(x)$. \\

\noindent We start from:
\begin{align*}
	N_X(\lambda x) &= \inf\{|\alpha | \, | \lambda x \in \alpha X\}
\end{align*}
But we know that $\lambda x \in \alpha X \Leftrightarrow \lambda x \in -\alpha X$, hence we can write:
\begin{align*}
	N_X(\lambda x) &= \inf\{|\alpha | \ | \ |\lambda| x \in \alpha X\}\\
	&= \inf\{|\lambda \alpha | \ | \ x \in \alpha X\}\\
	&= |\lambda|\inf\{|\alpha | \ | \ x \in \alpha X\}\\
	&= |\lambda|N_X(x)
\end{align*}
We have hence proved $N$ is homogeneous.

\subsubsection{}%Exercise 1.3.c
We want to show that $N$ is definite, that is $N_X(x) = 0 \, \Rightarrow \, x=0$. \\

\noindent We start from:
\begin{align*}
	& N_X(0) = \inf\{|\alpha | \, |  \, x \in \alpha X\} = 0\\
	\Rightarrow& \, x \in 0 \times X\\
	\Rightarrow& \, ||x|| \leq 0 \times M_X\\
	\Rightarrow& \, x = 0
\end{align*}
This concludes the proof.


\subsubsection{}%Exercise 1.3.d

We want to show that the triangular inequality is true for $N_X$.\\ 
We therefore take $x,y \in E$.
\begin{align*}
N_X(x+y) &= \inf\{|\alpha| \ | \ (x+y) \in \alpha X\}\\
&= \inf\left\{|\alpha| \ | \ (x+y) \in \frac{\alpha}{2} X+ \frac{\alpha}{2} X\right\}\\
&= \inf\left\{|\alpha| \ | \ x \in \frac{\alpha}{2}  X\right\} +  \textrm{inf}\left\{|\alpha| \ | \ y\in \frac{\alpha}{2}  X\right\}\\
&\leq \inf\{|\alpha| \ | \ x\in \alpha X\} + \textrm{inf}\{|\alpha| \ | \ y\in \alpha X\}\\
&= N_X(x) + N_X(y)
\end{align*}
This concludes the proof that the triangular inequality holds for $N_X$, and by extension, that $N_X$ is a norm.

\subsection{} %Exercise 1.4

\subsubsection{} %Exercise 1.4.a

We want to show that $ \forall x \in E, \, ||x|| \leq MN_X(x)$. In order to do this, we start from the fact that $X$ is bounded by $M$ and that therefore $X \subset \bar{B}(0,M)$. We therefore have $x \in N_X(x)\bar{X} \subset N_X(x)\bar{B}(0,M)$. Since $x \in N_X(x)\bar{B}(0,M)$ then $||x|| \leq MN_X(x)$. This concludes the proof that $|| \cdot ||$ is weaker than $N_X$.

\subsubsection{} %Exercise 1.4.b

We want to show that there exists an $\alpha$ such that for all $x$, $N_X(x) \leq \alpha ||x||$.
Since $0 \in \mathring{X}$, we can consider $r := sup\{r \, | \, B(0,r) \subset X \}$ with $r \neq 0$. We then have that for all $x$, $N_XB(0,r) \subset B(0,x)$. We therefore have that:
\begin{align*}
	N_Xr &\leq ||x||\\
	N_X &\leq \frac{1}{r}||x||
\end{align*}
This concludes the proof that $N_X$ is weaker than $|| \cdot ||$, and therefore also that they are equivalent.

\subsection{} %Exercise 1.5

We know that $x \in \bar{X}$ for the norm $N_X$ is equivalent to
\begin{align*}
	\exists (x_n)_{n \in \mathbb{N}} \in X^{\mathbb{N}}, (x_n) \text{ converges for }N_X \text{ to } x
\end{align*}
But we know that the norms $||\cdot ||$ and $N_X$ are equivalent, therefore if a sequence converges under $N_X$, then it must also converge under $|| \cdot ||$. Therefore:
\begin{align*}
	 x \in \bar{X}_{N_X} &\Leftrightarrow \exists (x_n)_{n \in \mathbb{N}} \in X^{\mathbb{N}},(x_n) \text{ converges for }|| \cdot || \text{ to } x\\
	&\Leftrightarrow x \in \bar{X}_{|| \cdot ||}
\end{align*}
We have therefore shown equivalence between the closure of $X$ for the two norms. This concludes the proof.
	


\section{Exercise 2}

\subsection{} %Exercise 2.1

Let us prove that $\bar{B}(0,1)$ is compact if and only if $S(0, 1) = \{ x\in E \ | \ N(x)=1 \}$ is compact. We will prove this by double implication. \\

\noindent Assume $\bar{B}(0,1) = \{ x \in E \ | \ N(x) \leq 1 \}$ is compact. \\
We know $S(0, 1) \subset \bar{B}(0,1)$ then $S(0,1)$ is compact provided it is closed as any closed subset of a compact set is compact. Let us now prove $S(0, 1)$ is closed by proving its complement is open. We have:
\begin{align*}
	(S(0,1))^c &= \{ x\in E \ | \ N(x) \neq 1 \} \\
	&=  \{ x\in E \ | \ N(x) < 1 \lor N(x) > 1 \} \\
	&= \{ x\in E \ | \ N(x) < 1 \} \cup \{ x\in E \ | \ N(x) > 1 \} \\
	&= B(0, 1) \cup (\bar{B}(0, 1))^c 
\end{align*}
We know $B(0,1)$ is open and as $\bar{B}(0, 1)$ is closed, $ (\bar{B}(0, 1))^c$ is open and hence $(S(0,1))^c$ is open as the union of open sets. Thus  $S(0, 1)$ is closed and therefore compact. \\

\noindent Let us now prove the converse statement. Assume $S(0, 1)$ is compact. \\

\noindent Consider the function 
\begin{align*} 
f  : \,  S(0,1) \times [0,1]  &\to  E \\
   (x,t)  &\mapsto  tx 
\end{align*} \\
Let us prove that $f$ is Lipschitz continuous. \\
Let $x, y \in S(0,1) $ and let $t_x, t_y \in [0,1]$. Then $N(t_xx-t_yy) \leq N(x-y)$ and thus $f$ is Lipschitz continuous. \\
Hence $f(S(0,1), [0,1])$ is compact. \\

\noindent Let us now prove that $f(S(0,1), [0,1]) = \bar{B}(0,1)$ by double inclusion.
\begin{itemize}
\item ''$\subseteq$'' 
Take $x \in S(0,1)$ and $t \in [0,1]$. \\
Then $N(f(x,t)) = N(tx) = |t| \, N(x) \leq N(x) = 1$ hence $f(x, t) \in \bar{B}(0,1)$. \\
 Therefore we have proved $f(S(0,1), [0,1]) \subseteq \bar{B}(0,1)$
\item ''$\supseteq$''
Take $x \in \bar{B}(0,1)$. We must prove that there exist $x' \in S(0,1)$ and $t' \in [0,1]$ such that $f(x', t') = x't' = x$. 
	\begin{itemize}
	\item If $x=0$:
	then we can set $t' :=0$ and let $x'$ be any element of $S(0,1)$. 
	We then get $f(x', t') = x't' = 0 = x$ which proves $x \in f(S(0,1), [0,1])$ 
	\item If $x \neq 0$:
	Let $t := \frac{1}{N(x)}$, then $N(tx) = |t| \, N(x) = 1$ thus we can set $x' := tx \in S(0,1)$. 
	We then set $t' := N(x) \in [0,1]$ so that $x = x't'$. 
	We hence get $f(x', t') = x$ which proves $x \in f(S(0,1), [0,1])$ 
	\end{itemize}
We therefore have proved that $\bar{B}(0,1) \subseteq S(0,1)$. 
\end{itemize}

\noindent Finally, $f(S(0,1), [0,1]) = \bar{B}(0,1)$ and hence $\bar{B}(0,1)$ is compact which finally concludes the proof. 


\subsection{} %Exercise 2.2

We first recall the definition of Lipschitz continuity for two normed vector spaces each endowed with their own norm, $(E,N_E), (F,N_F)$. The map $f$ is 1-Lipschitz continuous if:
\begin{align*}
	\forall x,y \in E, \, N_F\big( f(x) - f(y) \big) \leq N_E(x-y)
\end{align*}
In this case, we are considering the function $f:\, E \rightarrow \mathbb{R}, \, x \mapsto d(x,A) = inf\{ N(x-u) \, | \, u \in A \}$  with $A \subset E$. We want to show that the map $f$ is 1-Lipschitz continuous. That is to say that:
\begin{align*}
	\big| f(x) - f(y) \big| \leq N(x-y)
\end{align*}
First we consider an element $z_1 \in \bar{A}$ such that $d(y,z_1) = d(y,A)$. By the triangular inequality for distances, we have:
\begin{align*}
	d(x,z_1) \leq d(x,y) + d(y,z_1)
\end{align*}
However, we know that by definition $d(x,A) \leq d(x,z_1)$, and $d(y,z_1) = d(y,A)$, hence:
\begin{align*}
	d(x,A) &\leq d(x,y) + d(y,A)\\
	d(x,A) - d(y,A) &\leq d(x,y) \\
\end{align*}
We will now consider an element $z_2 \in \bar{A}$ such that $d(x,z_2) = d(x,A)$. By the same logic as above:
\begin{align*}
	d(y,z_2) \leq d(y,x) + d(x,z_2)
\end{align*}
However, we know that by definition $d(y,A) \leq d(y,z_2)$, and $d(x,z_2) = d(x,A)$. Also, we have that $d(x,y) = d(y,x)$. Hence:
\begin{align*}
	d(y,A) &\leq d(y,x) + d(x,A)\\
	-\big( d(x,A) - d(y,A) \big) &\leq d(x,y) \\
\end{align*}
By combining these two results, we have that:
\begin{align*}
	\big| d(x,A) - d(y,A) \big| &\leq d(x,y)\\
	\big| f(x) - f(y) \big| &\leq N(x-y)
\end{align*}
This concludes the proof of 1-Lipschitz continuity.

\subsection{} %Exercise 2.3

Let us first prove that $d(x, F) \leq N(x)$. \\
From the previous question, we have that $x \mapsto d(x, F)$ is Lipschitz-continuous so: $$\forall x, y \in F \  |d(x, F) -d(y, F)| \leq N(x - y)$$ Now let $y=0$ which is possible as $0 \in F$ because F is a vector subspace of E. We then get $|d(x, F) -d(0, F)| = |d(x, F)| = d(x, F) \leq N(x -0) = N(x)$ from which we get our result. \\

\noindent Let us now prove that $d(x, F) = \inf \{N(x-u) \,| \, u \in F, \, N(u) \leq 2N(x) \}$. We have from the previous question that since $F$ is a subset of $E$, $d(x, F) = \inf \{N(x-u) \,| \, u \in F\}$. We also know that $0 \in F$. Assume therefore that $N(u) > 2N(x)$, then $d(u,x) > N(x) = d(x,0)$. In other words, since zero is in $F$, we need not consider elements of F whose distance from $x$ will be greater than the distance between $x$ and $0$.

\subsection{} %Exercise 2.4

We want to show that $d(x,F)$ is attained at some $u_x\in F$. That is to say $\exists \, u_x \in F, d(x,u_x) = d(x,F)$. We therefore define a function
	\begin{align*}
	f: F &\to \mathbb{R}_+\\
	u &\mapsto N(x-u) = d(x,u)
	\end{align*}
We will also define a set $U := \{u \in F \, | \, N(u) \leq 2N(x)\}$.\\ 
\noindent We have from previous questions that $d(x,F) = \inf\{ N(x-u) \, | \, u \in U\} = \inf\{ f(u) \, | \, u \in U\}$. The necessary conditons for $d(x,F)$ to be attained are that $U$ is compact and $f$ is continuous.\\
\noindent We have by definition that $U$ is bounded, with $M = 2N(x)$. Snce $F$ is finite dimensional, it now suffices to prove that $U$ is closed to satisfy compactness. We will argue by contradiction, and assume that there exists a sequence $(x_n)_{n \in \mathbb{N}} \in U^{\mathbb{N}}$ converging to $l \not \in U$. Then for some $\epsilon > 0$, $\exists n \in \mathbb{N}, \, N(l-x_n) < \epsilon$.
\begin{align*}
	N(l) &= N(l-x_n+x_n) \\
	&\leq N(x_n -l) + N(x_n) \\
	&< \epsilon + N(x_n) \\
	&\leq  \epsilon + 2N(x)
\end{align*}
As $\epsilon \rightarrow 0$, we find that $l$ is in fact in $U$, showing by contradiction that $U$ is therefore closed.\\
\noindent We will now show that f is continuous. We will do this using the triangular inequality, taking $y, z \in F$. 
	\begin{align*}
		N(y-x) &= N(y-z+z-x) \leq N(y-z) + N(z-x) \Leftrightarrow \quad \, f(y) - f(z) \ \leq N(y-z)\\
		N(z-x) &= N(z-y+y-x) \leq N(y-z) + N(y-x) \Leftrightarrow -(f(y) - f(z)) \leq N(y-z)
	\end{align*}
We therefore have that $|f(y) - f(z)| \leq N(y-z)$ showing that $f$ is a continuous function.\\
\noindent Combining the compactness of $U$ and the continuity of $f$ concludes the proof that  finally gives the result that $d(x,F)$ is attained at some $u_x\in F$.

\subsection{} %Exercise 2.5
We want to show that for an element $y:=x-u_x \in E$, 
\begin{align*}
	\sup _{N(y) = 1} d(y,F) \in \{ 0,1\}
\end{align*}
We will first consider the case where $F = E$, where $d(y,F)$ can take no other value than 0. In that case we trivially have
\begin{align*}
	\sup _{N(y) = 1} d(y,F) = 0 \in \{ 0,1\}
\end{align*}
We now consider the case where $F \subsetneq E$. We have from a previous question that $d(y,F) \leq N(y) = 1$. We take $u_y \in F$ to be the element of F for which $d(y,F)$ is attained. We therefore have that $\forall u \in F, \, N(y - u_y) \leq N(y - u)$. Along the same principle, we have that $\forall u \in F, \, N(x - u_x) \leq N(x - u)$. Since $u_x + u_y \in F$, $N(x - u_x) \leq N\big( x - (u_x + u_y)\big)$.\\
\noindent We now use the initial statement that $\forall x \in E, \, d(x,f) \leq N(x)$, to say that $N(y- u_y) \leq N(y)$, using the definition of $y$ to rewrite this as $N\big( x - (u_x + u_y)\big) \leq N(x - u_x)$.\\
\noindent The two sides of the inequality conclude that $N\big( x - (u_x + u_y)\big) = N(x - u_x)$, that is to say $N(y- u_y) = N(y) = 1$. This concludes the proof that $\sup _{N(y) = 1} d(y,F) \in \{ 0,1\}$.

\subsection{} %Exercise 2.6

Let us construct this infinite sequence by induction. \\

\noindent Start by $x_0 \in S(0,1)$, then $N(x_0)=1$. \\
Let us prove $Span(x_0) \supsetneq S(0,1)$ by contradiction. Assume $x_0$ spans $S(0,1)$ then it would also span $E$ which is a contradiction as $E$ is of infinite dimension. \\

\noindent Then take $x_1 \in S(0,1)$ such that $N(x_1-x_0)=1$, we also have $N(x_1)=1$ \\
By the same arguments and as $\dim E > 1$, $(x_0 - x_1)$ doesn't span $S(0,1)$ \\

\noindent Then by induction, as $\dim E = + \infty$ we can construct an infinite sequence $(x_i)_{i \in \mathbb{N}}$ that spans $E$ such that $N(x_n - x_m) = 1$ if $n \neq m$ and $N(x_n) = 1$ for $n \geq 0$.\\

\noindent One cannot then extract a subsequence that converges in $S(0,1)$, thus $S(0,1)$ is not compact. Using the first question of the exercise, we therefore have that, $\Bar{B}(0,1)$ is not compact. 


\section{Exercise 3}
\subsection{} %Exercise 3.1
We will prove by contradiction that all compact sets are precompact. \\

\noindent Let F be a compact and non-precompact subset of E.\\
Let $I \subset \mathbb{N}$, such that $\exists (x_i)_{i \in I}$ such that $F\subset\bigcup\limits_{i\in I}B(x_i, \epsilon)$ 
Let us construct the sequence $(u_n)_{n \in I} \in F^{\mathbb{N}}$, such that $u_i \in B(x_i,\epsilon) \setminus \bigcup \limits_{j\in I, j \neq i}B(x_j, \epsilon) \subset F$ \\
We have, $\forall i,j \in I,i \neq j$, $N(u_i-u_j) \geq \frac{1}{2} \min_{i,j \in I}\big( N(x_i-x_j)\big)$.\\
Thus $\forall (u_n)_{n \in I}, \forall \phi \text{ strictly increasing}, (u_{\phi(n)})$ does not converge.\\
Thus $(u_n)_{n \in I}$ does not have an adherence value and F is therefore not compact. This is a contradiction with the fact that F is compact.\\
Therefore, all compact sets are precompact.\\


\section{Exercise 4}
\subsection{} %Exercise 4.1
We take as our example:
\begin{align*}
	H &= \mathcal{C}^2\big( [0,1] , \mathbb{C} \big)\\
	N &= ||\cdot || _{\infty} \big|_{H} : f \mapsto \sup_{t \in [0,1]} |f(t)|
\end{align*}
We first prove that $H \subset \mathcal{C}^1\big( [0,1] , \mathbb{C} \big)$. We know that all functions which are of class $\mathcal{C}^2$ are also $\mathcal{C}^1$, and the functions are still mapping from $[0,1]$ to $\mathbb{C}$. Thus $H \subset \mathcal{C}^1\big( [0,1] , \mathbb{C} \big)$.\\

\noindent We now prove that $(H, N)$ has infinite dimension. \\
We know $\mathbb{C}[X] \subset \mathcal{C}^2\big( [0,1] , \mathbb{C} \big)$ as complex polynomials are of class $\mathcal{C}^2$. \\
But we also know $\mathbb{C}[X]$ has infinite dimension as its basis $(1, X, X^2, X^3, ...)$ is composed of infinitely many vectors which finally proves that $(H, N)$ has infinite dimension. 

\end{document}
